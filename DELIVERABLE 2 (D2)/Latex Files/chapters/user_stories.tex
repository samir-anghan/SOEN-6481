\section{User stories for calculator system}

%%%%%%%%%%%%%%%%%%%% Table No: 1 starts here %%%%%%%%%%%%%%%%%%%%
\hspace{1cm}
\begin{center}
\begin{tabular}{ | m{2cm} | m{12cm} | } 

 \hline
 \multicolumn{2}{|c|}{\textbf{User Story 1}} \\

\hline
\textbf{Id} & US1 \\ 

\hline
\textbf{User Story Statement} & As a mathematician, I want to calculate the value of the silver ratio number up to given certain decimal places, so that I can see what the number is up to certain decimal places. \\ 

\hline
\textbf{Acceptance Criteria} & Given that I need to calculate the value of silver ratio number having 10 digits after the decimal point,

When I perform an operation by providing 10 as a number of digits I want after the decimal point,

I should see 2.4142135623 as an answer.\\ 

\hline
\textbf{Priority} & Must have \\ 

\hline
\textbf{Constraint} & Usability-specific: A calculator user can calculate the value of the silver ratio number having a maximum of 20 digits after the decimal point.\\ 

\hline
\textbf{Estimated Story Points} & 6 \\ 
\hline

\end{tabular}
\end{center}
%%%%%%%%%%%%%%%%%%%% Table No: 1 ends here %%%%%%%%%%%%%%%%%%%%

%%%%%%%%%%%%%%%%%%%% Table No: 2 starts here %%%%%%%%%%%%%%%%%%%%
\hspace{1cm}
\begin{center}
\begin{tabular}{ | m{2cm} | m{12cm} | } 

 \hline
 \multicolumn{2}{|c|}{\textbf{User Story 2}} \\

\hline
\textbf{Id} & US2 \\ 

\hline
\textbf{User Story Statement} &  As a mathematician, I want to calculate an area of a regular octagon with given side length, so that I can see what the area is for a given side length.\\ 

\hline
\textbf{Acceptance Criteria} &  
Given that I need to calculate an area of a regular octagon with a side length of 8,

When I perform an operation by providing 8 as a side length of an octagon,

Then I should see 309.02 as an answer.\\ 

\hline
\textbf{Priority} & Must have \\ 

\hline
\textbf{Constraint} & NIL\\ 

\hline
\textbf{Estimated Story Points} & 6 \\
\hline

\end{tabular}
\end{center}
%%%%%%%%%%%%%%%%%%%% Table No: 2 ends here %%%%%%%%%%%%%%%%%%%%

%%%%%%%%%%%%%%%%%%%% Table No: 3 starts here %%%%%%%%%%%%%%%%%%%%
\hspace{1cm}
\begin{center}
\begin{tabular}{ | m{2cm} | m{12cm} | } 

 \hline
 \multicolumn{2}{|c|}{\textbf{User Story 3}} \\

\hline
\textbf{Id} & US3 \\ 

\hline
\textbf{User Story Statement} &  As a mathematician, I want to store a calculated value of the silver ratio number in memory, so that I can use it later.\\ 

\hline
\textbf{Acceptance Criteria} & Given that I have already calculated the value of the silver ratio number having 10 digits after the decimal point,

When I press ``M in'' key,

Then the number of 2.4142135623 should be stored in memory,

And the status bar on the display should show ``M'',

And the calculator should allow me to do the next operation.\\ 

\hline
\textbf{Priority} & Must have \\ 

\hline
\textbf{Constraint} & NIL\\ 

\hline
\textbf{Estimated Story Points} & 6 \\ 
\hline

\end{tabular}
\end{center}
%%%%%%%%%%%%%%%%%%%% Table No: 3 ends here %%%%%%%%%%%%%%%%%%%%

%%%%%%%%%%%%%%%%%%%% Table No: 4 starts here %%%%%%%%%%%%%%%%%%%%
\hspace{1cm}
\begin{center}
\begin{tabular}{ | m{2cm} | m{12cm} | } 

 \hline
 \multicolumn{2}{|c|}{\textbf{User Story 4}} \\

\hline
\textbf{Id} & US4 \\ 

\hline
\textbf{User Story Statement} & \\ 

\hline
\textbf{Acceptance Criteria} & \\ 

\hline
\textbf{Priority} & Must have \\ 

\hline
\textbf{Constraint} & NIL\\ 

\hline
\textbf{Estimated Story Points} & 6 \\ 
\hline

\end{tabular}
\end{center}
%%%%%%%%%%%%%%%%%%%% Table No: 4 ends here %%%%%%%%%%%%%%%%%%%%

%%%%%%%%%%%%%%%%%%%% Table No: 5 starts here %%%%%%%%%%%%%%%%%%%%
\hspace{1cm}
\begin{center}
\begin{tabular}{ | m{2cm} | m{12cm} | } 

 \hline
 \multicolumn{2}{|c|}{\textbf{User Story 5}} \\

\hline
\textbf{Id} & US5 \\ 

\hline
\textbf{User Story Statement} & As a mathematician, I want to add a certain number to the value of the silver ratio number, so that I can see what their total is. \\ 

\hline
\textbf{Acceptance Criteria} & Given that I have two numbers 5 and the silver ratio number,

When I perform addition on them,

Then I should see the sum as 7.4142135623. \\ 

\hline
\textbf{Priority} & Must have \\ 

\hline
\textbf{Constraint} & The addition expression should use the number 2.4142135623 as a value of the silver ratio number, which has exactly 10 digits after the decimal point.\\ 

\hline
\textbf{Estimated Story Points} & 6 \\ 
\hline

\end{tabular}
\end{center}
%%%%%%%%%%%%%%%%%%%% Table No: 5 ends here %%%%%%%%%%%%%%%%%%%%

%%%%%%%%%%%%%%%%%%%% Table No: 6 starts here %%%%%%%%%%%%%%%%%%%%
\hspace{1cm}
\begin{center}
\begin{tabular}{ | m{2cm} | m{12cm} | } 

 \hline
 \multicolumn{2}{|c|}{\textbf{User Story 6}} \\

\hline
\textbf{Id} & US6 \\ 

\hline
\textbf{User Story Statement} & As a mathematician, I want to subtract a certain number from the value of the silver ratio number, so that I can see what the difference between them is.\\ 

\hline
\textbf{Acceptance Criteria 1} & Given that I have two numbers 2 and the silver ratio number, 

When I subtract 2 from the silver ratio number,

Then I should see the difference as 0.4142135623.\\ 

\hline
\textbf{Acceptance Criteria 2} & Given that I have two numbers 10 and the silver ratio number, 

When I subtract 2 from the silver ratio number,

Then I should see the difference as -7.5857864377 which is a negative number.\\

\hline
\textbf{Priority} & Must have \\ 

\hline
\textbf{Constraint} & The subtraction expression should use the number 2.4142135623 as a value of the silver ratio number, which has exactly 10 digits after the decimal point.\\ 

\hline
\textbf{Estimated Story Points} & 6 \\ 
\hline

\end{tabular}
\end{center}
%%%%%%%%%%%%%%%%%%%% Table No: 6 ends here %%%%%%%%%%%%%%%%%%%%

%%%%%%%%%%%%%%%%%%%% Table No: 7 starts here %%%%%%%%%%%%%%%%%%%%
\hspace{1cm}
\begin{center}
\begin{tabular}{ | m{2cm} | m{12cm} | } 

 \hline
 \multicolumn{2}{|c|}{\textbf{User Story 7}} \\

\hline
\textbf{Id} & US7 \\ 

\hline
\textbf{User Story Statement} & As a mathematician, I want to multiply a certain number with the value of the silver ratio number, so that I can see what their product is.\\ 

\hline
\textbf{Acceptance Criteria 1} & Given that I have two numbers 5 and the silver ratio number, 

When I multiply 5 with the silver ratio number,

Then I should see the product as 12.0710678115.\\ 

\hline
\textbf{Acceptance Criteria 2} & Given that I have two numbers 0 and the silver ratio number, 

When I multiply 0 with the silver ratio number,

Then I should see the product as 0.\\ 

\hline
\textbf{Acceptance Criteria 3} & Given that I have two numbers 1 and the silver ratio number, 

When I multiply 1 with the silver ratio number,

Then I should see the product as 2.4142135623 which is the same as the silver ratio number.\\ 

\hline
\textbf{Priority} & Must have \\ 

\hline
\textbf{Constraint} & The multiplication expression should use the number 2.4142135623 as a value of the silver ratio number, which has exactly 10 digits after the decimal point.\\ 

\hline
\textbf{Estimated Story Points} & 6 \\ 
\hline

\end{tabular}
\end{center}
%%%%%%%%%%%%%%%%%%%% Table No: 7 ends here %%%%%%%%%%%%%%%%%%%%

%%%%%%%%%%%%%%%%%%%% Table No: 8 starts here %%%%%%%%%%%%%%%%%%%%
\hspace{1cm}
\begin{center}
\begin{tabular}{ | m{2cm} | m{12cm} | } 

 \hline
 \multicolumn{2}{|c|}{\textbf{User Story 8}} \\

\hline
\textbf{Id} & US8 \\ 

\hline
\textbf{User Story Statement} & As a mathematician, I want to divide a certain number by the value of the silver ratio number, so that I can see what the quotient is.\\ 

\hline
\textbf{Acceptance Criteria 1} & Given that I have two numbers 10 and the silver ratio number, 

When I divide 10 by the silver ratio number,

Then I should see the quotient as 4.14213562386.\\ 

\hline
\textbf{Acceptance Criteria 2} & Given that I have two numbers 0 and the silver ratio number, 

When I divide 0 by the silver ratio number,

Then I should see the quotient as 0.\\ 

\hline
\textbf{Priority} & Must have \\ 

\hline
\textbf{Constraint} & The division expression should use the number 2.4142135623 as a value of the silver ratio number, which has exactly 10 digits after the decimal point.\\ 

\hline
\textbf{Estimated Story Points} & 6 \\ 
\hline

\end{tabular}
\end{center}
%%%%%%%%%%%%%%%%%%%% Table No: 8 ends here %%%%%%%%%%%%%%%%%%%%

%%%%%%%%%%%%%%%%%%%% Table No: 9 starts here %%%%%%%%%%%%%%%%%%%%
\hspace{1cm}
\begin{center}
\begin{tabular}{ | m{2cm} | m{12cm} | } 

 \hline
 \multicolumn{2}{|c|}{\textbf{User Story 9}} \\

\hline
\textbf{Id} & US9 \\ 

\hline
\textbf{User Story Statement} & As a mathematician, I want to divide the value of the silver ratio number by a certain number, so that I can see what the quotient is.\\ 

\hline
\textbf{Acceptance Criteria 1} & Given that I have two numbers the silver ratio number and 100,

When I divide the silver ratio number by 100,

Then I should see the quotient as 0.02414213562.\\ 

\hline
\textbf{Acceptance Criteria 2} & Given that I have two numbers the silver ratio number and 0,

When I divide the silver ratio number by 0,

Then I should see the quotient as “infinity”.\\ 

\hline
\textbf{Priority} & Must have \\ 

\hline
\textbf{Constraint} & The division expression should use the number 2.4142135623 as a value of the silver ratio number, which has exactly 10 digits after the decimal point.\\ 

\hline
\textbf{Estimated Story Points} & 6 \\ 
\hline

\end{tabular}
\end{center}
%%%%%%%%%%%%%%%%%%%% Table No: 9 ends here %%%%%%%%%%%%%%%%%%%%