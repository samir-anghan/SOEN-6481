\documentclass[12pt, letterpaper]{article}
\usepackage{amsmath}
\usepackage{latexsym}
\usepackage{amsfonts}
\usepackage[normalem]{ulem}
\usepackage{array}
\usepackage{amssymb}
\usepackage{graphicx}
\usepackage[backend=biber,
style=numeric,
sorting=none,
isbn=false,
doi=false,
url=false,
]{biblatex}\addbibresource{bibliography.bib}

\usepackage{subfig}
\usepackage{wrapfig}
\usepackage{txfonts}
\usepackage{wasysym}
\usepackage{enumitem}
\usepackage{adjustbox}
\usepackage{ragged2e}
\usepackage[svgnames,table]{xcolor}
\usepackage{tikz}
\usepackage{longtable}
\usepackage{changepage}
\usepackage{setspace}
\usepackage{hhline}
\usepackage{multicol}
\usepackage{tabto}
\usepackage{float}
\usepackage{multirow}
\usepackage{makecell}
\usepackage{fancyhdr}
\usepackage[toc,page]{appendix}
\usepackage[hidelinks]{hyperref}
\usetikzlibrary{shapes.symbols,shapes.geometric,shadows,arrows.meta}
\tikzset{>={Latex[width=1.5mm,length=2mm]}}
\usepackage{flowchart}\usepackage[paperheight=11.0in,paperwidth=8.5in,left=1.0in,right=1.0in,top=1.0in,bottom=1.0in,headheight=1in]{geometry}
\usepackage[utf8]{inputenc}
\usepackage[T1]{fontenc}
\TabPositions{0.5in,1.0in,1.5in,2.0in,2.5in,3.0in,3.5in,4.0in,4.5in,5.0in,5.5in,6.0in,}

\urlstyle{same}

\setcounter{tocdepth}{5}
\setcounter{secnumdepth}{5}

\setlistdepth{9}
\renewlist{enumerate}{enumerate}{9}
		\setlist[enumerate,1]{label=\arabic*)}
		\setlist[enumerate,2]{label=\alph*)}
		\setlist[enumerate,3]{label=(\roman*)}
		\setlist[enumerate,4]{label=(\arabic*)}
		\setlist[enumerate,5]{label=(\Alph*)}
		\setlist[enumerate,6]{label=(\Roman*)}
		\setlist[enumerate,7]{label=\arabic*}
		\setlist[enumerate,8]{label=\alph*}
		\setlist[enumerate,9]{label=\roman*}

\renewlist{itemize}{itemize}{9}
		\setlist[itemize]{label=$\cdot$}
		\setlist[itemize,1]{label=\textbullet}
		\setlist[itemize,2]{label=$\circ$}
		\setlist[itemize,3]{label=$\ast$}
		\setlist[itemize,4]{label=$\dagger$}
		\setlist[itemize,5]{label=$\triangleright$}
		\setlist[itemize,6]{label=$\bigstar$}
		\setlist[itemize,7]{label=$\blacklozenge$}
		\setlist[itemize,8]{label=$\prime$}

\setlength{\topsep}{0pt}\setlength{\parindent}{0pt}

\renewcommand{\arraystretch}{1.3}


\title{An interview of a potential user of irrational numbers}
\author{Samir Anghan: 40040308}
\date{Friday, July 12, 2019}

\begin{document}

\maketitle

\section{Interviewer}
Samir Anghan\par
Concordia University Gina Cody School of Engineering and Computer Science\par

\section{Interviewee}
Mehul Patel\par
Electronics circuit design engineer, Rambus Chip Technologies (India) Private Limited.\par
M.tech in Electrical Engineering with specialization Electronic Systems from 'Indian Institute of Technology Bombay - India'\par

\section{The rationale for choosing Mehul Patel as an interviewee}

\begin{justify}
Mehul Patel is an Electronics Circuit Design Engineer with a background of mathematics. An electronics circuit design engineer is a person who uses mathematics in their everyday tasks at his/her work. My interviewee, Mehul Patel, also confirmed that almost all electronics circuit design engineers do use of mathematics and an electric circuit simulator using MATHEMATICA Software. They often need to provide numerical values to circuit parameters. This brings to a conclusion that a person who is an electronics circuit design engineer is usually close to the use of mathematics and all numbers including irrational numbers. Hence, I believe that my interviewee is a potential user of given ETERNITY: NUMBERS.
\end{justify}\par


\section{Interview questions and responses}

\begin{table}[H]
 			\centering
\begin{tabular}{p{0.72in}p{5.37in}}
\hline

\multicolumn{1}{|p{0.72in}}{\textbf{Question}} & 
\multicolumn{1}{|p{5.37in}|}{As an engineer, which of the following irrational numbers you use or ever used in your everyday tasks at your work? 
\vspace{\baselineskip}
 \par Number 1: Champernowne Constant \par Number 2: Euler’s Number \par Number 3: Gaussian Integral \par Number 4: Gelfond’s Constant \par Number 5: Golden Ratio \par Number 6: Liouville Constant \par Number 7: Natural Logarithm of 2 \par Number 8: Pi \par Number 9: Silver Ratio \par Number 10: Universal Parabolic Constant \par Number 11: Plastic Number \par Number 12: Hilbert Number 
 \vspace{\baselineskip}
 \par Please mention here the number (e.g. Number 1, 10, 11) here: \par } \\
\hhline{--}

\multicolumn{1}{|p{0.72in}}{\textbf{Response}} & 
\multicolumn{1}{|p{5.37in}|}{Number - 2,3,5,7,8,9,10,11, I have used.} \\
\hhline{--}

\end{tabular}
 \end{table}


\begin{table}[H]
 			\centering
\begin{tabular}{p{0.72in}p{5.37in}}
\hline

\multicolumn{1}{|p{0.72in}}{\textbf{Question}} & 
\multicolumn{1}{|p{5.37in}|}{Do you know any real-life application that uses irrational numbers?  \par Answer in $``$Yes$"$  or $``$No$"$ . If yes, please mention. \par } \\
\hhline{--}

\multicolumn{1}{|p{0.72in}}{\textbf{Response}} & 
\multicolumn{1}{|p{5.37in}|}{There are lots of real-life application that uses irrational numbers. Like PI is used in almost all geometric calculation in real life. e is used in compound interest.} \\
\hhline{--}

\end{tabular}
 \end{table}



\begin{table}[H]
 			\centering
\begin{tabular}{p{0.71in}p{5.38in}}
\hline

\multicolumn{1}{|p{0.71in}}{\textbf{Question}} & 
\multicolumn{1}{|p{5.38in}|}{If you ever want to use a calculator that computes the value of certain irrational numbers, what other additional functionalities from below list you would like to have in that calculator. 
\vspace{\baselineskip}
\par Function 1: Calculate the value of irrational number up to given certain decimal places. \par Function 2: Addition, Subtraction, Multiplication, Division of the Irrational Numbers. \par Function 3: Classify the given number whether it is rational or irrational. \par Function 4: Other (Please describe) \par } \\
\hhline{--}

\multicolumn{1}{|p{0.71in}}{\textbf{Response}} & 
\multicolumn{1}{|p{5.38in}|}{Function 1 to 3, I would like to prefer. Apart from these, below functions also I want to prefer \par \begin{itemize}
	\item can enter the symbol of some generally used irrational number (like PI, e, silver ratio, golden ratio) \par 	\item common root calculation (squared, cubed) and trigonometric calculation \par 	\item equation having irrational numbers build up the facility \par 	\item storage of some calculated irrational parameter
\end{itemize} \par } \\
\hhline{--}

\end{tabular}
 \end{table}
 
 

\begin{table}[H]
 			\centering
\begin{tabular}{p{0.7in}p{5.39in}}
\hline

\multicolumn{1}{|p{0.7in}}{\textbf{Question}} & 
\multicolumn{1}{|p{5.39in}|}{Are you using any existing mathematical software for any required mathematical operations on irrational numbers?  \par Answer in $``$Yes$"$  or $``$No$"$ . If $``$Yes$"$, please provide information. \par } \\
\hhline{--}

\multicolumn{1}{|p{0.7in}}{\textbf{Response}} & 
\multicolumn{1}{|p{5.39in}|}{Yes, I have used MATLAB and MATHEMATICA software for any required mathematical operation on irrational numbers during my education span.} \\
\hhline{--}

\end{tabular}
 \end{table}



\begin{table}[H]
 			\centering
\begin{tabular}{p{0.7in}p{5.39in}}
\hline

\multicolumn{1}{|p{0.7in}}{\textbf{Question}} & 
\multicolumn{1}{|p{5.39in}|}{Silver Ratio ($ \delta $s) is an irrational number, whose value is one plus the square root of 2 and is approximately 2.4142135623. Have you ever used Silver Ratio during your education or at your work?  \par Answer in$``$Yes$"$  or $``$No$"$ . If $``$Yes$"$, please provide information on why or how you used the Silver Ratio number. \par } \\
\hhline{--}

\multicolumn{1}{|p{0.7in}}{\textbf{Response}} & 
\multicolumn{1}{|p{5.39in}|}{Not practically, but just because of my curiosity in this magic number (silver ratio), I have studied a little about this silver ratio during my education span.} \\
\hhline{--}

\end{tabular}
 \end{table}



\begin{table}[H]
 			\centering
\begin{tabular}{p{0.69in}p{5.4in}}
\hline

\multicolumn{1}{|p{0.69in}}{\textbf{Question}} & 
\multicolumn{1}{|p{5.4in}|}{The area of a regular octagon with side length of $a$ can be calculated by following formula which uses Silver Ratio. \par 
\[ A = 2 (\sqrt{2} + 1) {a}^2 \]
 \par Here the value of the square root of 2 is 1.4142135623730951$ \ldots $  (no finite number of digits). \par According to you, up to what number of certain decimal places, the value of the \(\sqrt{2} \) should be used in the above formula to calculate the area of a regular octagon? \par } \\
\hhline{--}

\multicolumn{1}{|p{0.69in}}{\textbf{Response}} & 
\multicolumn{1}{|p{5.4in}|}{It depends on how accurate the area number you want. Generally, a standard scientific calculator uses around 9-10 digit after the decimal point.} \\
\hhline{--}

\end{tabular}
 \end{table}

\section{Analysis of an interview}
After having an interview with Mehul Patel, an electronics circuit design engineer, I came to discover a few things regarding the irrational numbers. An interviewee is currently an employee at Rambus Chip Technologies (India) Private Limited and had M.tech in Electrical Engineering with specialization Electronic Systems. He has a passion for mathematics. From the responses of the question, I conclude the following things:\par

\begin{itemize}
	\item An interviewee has used or studied many irrational numbers but mostly during his education period.\par

	\item An interviewee is aware of some real-life application that uses irrational numbers.\par

	\item An interviewee suggested some functions to include in the Calculator (e.g. storage of some calculated irrational parameter).\par

	\item An interviewee has some knowledge of Silver Ratio ($ \delta $s) number, but he has not used the number practically yet.\par

	\item An interviewee suggested to consider up to 9-10 digits after the decimal point of an irrational number for any applications that use the value of an irrational number.
\end{itemize}\par

\printbibliography
\end{document}