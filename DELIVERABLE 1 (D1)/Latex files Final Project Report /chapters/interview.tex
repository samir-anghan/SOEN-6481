\section{Interviewer}
Samir Anghan, is a student at Concordia University Gina Cody School of Engineering and Computer Science.\par

\section{Interviewee}
Mehul Patel, is an Electronics circuit design engineer at Rambus Chip Technologies (India) Private Limited. He is graduated in Electrical Engineering with specialization Electronic Systems from 'Indian Institute of Technology Bombay - India'\par

\section{The rationale for choosing interviewee}
Mehul Patel is an Electronics Circuit Design Engineer with a background of mathematics. An electronics circuit design engineer is a person who uses mathematics in their everyday tasks at his/her work. My interviewee, Mehul Patel, also confirmed that almost all electronics circuit design engineers do use of mathematics and an electric circuit simulator using MATHEMATICA Software. They often need to provide numerical values to circuit parameters. This brings to a conclusion that a person who is an electronics circuit design engineer is usually close to the use of mathematics and all numbers including irrational numbers. Hence, I believe that my interviewee is a potential user of given ETERNITY: NUMBERS.

\section{Interview questions and responses}
\textbf{Question 1:} As an engineer, which of the following irrational numbers you use or ever used in your everyday tasks or during your work?
\begin{enumerate}
  \item Champernowne Constant
  \item Euler’s Number
  \item Gaussian Integral
  \item Gelfond’s Constant
  \item Golden Ratio
  \item Liouville Constant
  \item Natural Logarithm of 2
  \item Pi 
  \item Silver Ratio
  \item Universal Parabolic Constant
  \item Plastic Number
  \item Hilbert Number
\end{enumerate} 
\textbf{Response: } Champernowne Constant, Universal Parabolic Constant and Plastic Number.\\ \\
\textbf{Question 2:} Do you know any real-life application that uses irrational numbers? \\
Answer in “Yes” or “No”. If yes, please mention.\\ \\
\textbf{Response:} There are lots of real-life application that uses irrational numbers. Like PI is used in almost all geometric calculation in real life. e is used in compound interest. \\ \\
\textbf{Question 3:} If you ever want to use a calculator that computes the value of certain irrational numbers, what other additional functionality from below list you would like to have in that calculator.\\ \\
    Function 1: Calculate the value of irrational number up to given certain decimal places.\\
    Function 2: Addition, Subtraction, Multiplication, Division of the Irrational Numbers.\\
    Function 3: Classify the given number whether it is rational or irrational.\\
    Function 4: Other (Please describe) \\ \\
\textbf{Response:} I would like to have Function 1 to 3. I would also like to have below functions:
\begin{enumerate}
\item I could enter the symbol of some usually used irrational number such as Pi, euler's number.
\item Common root calculation (squared, cubed etc.) and trigonometric calculation.
\item Equation having irrational numbers build up the facility storage of some calculated irrational parameter.
\end{enumerate} 
\textbf{Question 4:} Are you using any existing mathematical software for any required mathematical operations on irrational numbers? \\
Answer “Yes” or “No”. If “Yes”, please provide information. \\ \\
\textbf{Response:} Yes, I have used MATLAB and MATHEMATICA software for any required mathematical operation on irrational numbers during my education period. \\ \\
\textbf{Question 5:} Silver Ratio ($ \delta $s) is an irrational number, whose value is one plus the square root of 2 and is approximately 2.4142135623. Have you ever used Silver Ratio during your education or at your work? \\
Answer in$``$Yes$"$  or $``$No$"$ . If $``$Yes$"$, please provide information on why or how you used the Silver Ratio number. \\ \\
\textbf{Response:} Not practically, but just because of my curiosity in this magic number (silver ratio), I have studied a little about this silver ratio number during my education period. \\ \\
\textbf{Question 6:} The area of a regular octagon with side length of $a$ can be calculated by following formula which uses Silver Ratio. 
\[ A = 2 (\sqrt{2} + 1) {a}^2 \]
Here the value of the square root of 2 is 1.4142135623730951$ \ldots $  (no finite number of digits). \\ According to you, up to what number of certain decimal places, the value of the \(\sqrt{2} \) should be used in the above formula to calculate the area of a regular octagon? \\ \\
\textbf{Response:} It depends on how accurate the area number you want. Generally, a standard scientific calculator uses around 9-10 digit after the decimal point. \\ \\
\textbf{Question 7:} Do you know any other possible uses of Silver Ratio number, by itself, or in combination with other numbers? (e.g. Silver Ratio can be used to calculate the area of a regular octagon) \\ Answer “Yes” or “No”. If “Yes”, please provide information. \\ \\
\textbf{Response:} Yes, the Silver ratio can be connected to the trigonometric ratio for Pi/8 value. 

\section{Analysis of an interview}
After having an interview with Mehul Patel, an electronics circuit design engineer, I came to discover a few things regarding the irrational numbers. An interviewee is currently an employee at Rambus Chip Technologies (India) Private Limited and had M.tech in Electrical Engineering with specialization Electronic Systems. He has a passion for mathematics. From the responses of the question, I conclude the following things:\par

\begin{itemize}
	\item An interviewee has used or studied many irrational numbers but mostly during his education period.\par

	\item An interviewee is aware of some real-life application that uses irrational numbers.\par

	\item An interviewee suggested some functions to include in the Calculator (e.g. storage of some calculated irrational parameter).\par

	\item An interviewee has some knowledge of Silver Ratio ($ \delta $s) number, but he has not used the number practically yet.\par

	\item An interviewee suggested to consider up to 9-10 digits after the decimal point of an irrational number for any applications that use the value of an irrational number.
\end{itemize}\par







