\documentclass[letterpaper]{article}
\usepackage[utf8]{inputenc}
\usepackage[margin=0.5in]{geometry}
\usepackage{amsmath}
\usepackage{hyperref}
\hypersetup{
    colorlinks=true,
    linkcolor=blue,
    filecolor=magenta,      
    urlcolor=blue,
}
\urlstyle{same}

\title{The Silver Ratio: An irrational mathematical constant}
\author{Samir Anghan: 40040308}
\date{Friday, July 5, 2019}

\begin{document}

\maketitle

\section{Introduction}
The silver ratio (also known as silver mean or silver constant) is an irrational mathematical constant whose value is approximately 2.4142135623. It is denoted by the greek letter $\delta$s. \\ \\
\textbf{\underline{Definition} [Silver Ratio] [Wikipedia]}. In mathematics, two quantities are in the silver ratio if the ratio of the sum of the smaller and twice the larger of those quantities, to the larger quantity, is the same as the ratio of the larger one to the smaller one. \\ \\
\textbf{Explanation:} \\
Let's say $n1$ and $n2$ are given two numbers where $ n1 > n2 $. The ratio $n1/n2$ of both the numbers is silver ratio if it equals to the ratio of the sum of the smaller number $n1$ and twice the larger number $n2$ to the larger number $n2$. \\
So, $n1/n2$ is the silver ratio if: \\

\[\frac{n1}{n2}=\frac{2n1 + n2}{n1} \]
\section{Characteristics}
\begin{enumerate}
  \item Pell number sequence (1, 2, 5, 12, 29...) tends to the silver ratio. In other words, ratio between two consecutive numbers from the Pell number sequence tends to the silver ratio. If we calculate the ratios of two consecutive Pell numbers \( \frac{Pn}{Pn-1} \), we get:
  \[  \delta s = \sqrt{2} + 1 \]
  
  \item If we draw a rectangle whose sides have ratio same as the silver ratio, that is \( (\sqrt{2} + 1) + 1 : 1\), it is called \emph{Silver Rectangle}. \\
Let's draw a large silver rectangle. Now if we remove the largest possible square from the drawn silver rectangle, it will yield a silver rectangle of the other kind, removing once again the largest possible square from it, will again yield an another silver rectangle. Repeating the process will always give us a silver rectangle (of course smaller silver rectangle each time).

 \item There is a relation between the silver ratio and the octagon. That is, in a regular octagon, the ratio between the orthogonal diagonal to a side is the silver ratio.
\end{enumerate}

\section{Usage Examples}
\begin{enumerate}
  \item The paper sizes under ISO 216 are rectangles, which has a proportion ratio of \( 1:\sqrt{2} \). This ratio is same as the silver ratio.
  \item The silver ratio is used in classical architectures and arts. For example, The architectures in the temples of Japan. \\
\end{enumerate}


\textbf{References} 
\begin{enumerate}
 \item “Silver Ratio.” Wikipedia, Wikimedia Foundation, 18 May 2019, \url{https://en.wikipedia.org/wiki/Silver_ratio}
 \item History of Mathematics. Prof. Shanyu Ji, \url{https://www.math.uh.edu/~shanyuji/History/2016/2016-2-3.pdf}
\end{enumerate}
\end{document}
